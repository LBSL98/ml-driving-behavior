\documentclass[12pt]{article}
\usepackage{sbc-template}
\usepackage[brazil]{babel}
\usepackage[utf8]{inputenc}
\usepackage{graphicx,url}
\usepackage{amsmath,amssymb}
\usepackage{booktabs}
\sloppy

\title{Classificação de Comportamento de Direção com Sensores de Smartphone:\\
UAH-DriveSet, Avaliação Cross-Driver e Calibração de Limiar por Rota}

\author{Matheus Reis\inst{1}}
\address{UTFPR\\ \email{seu.email@alunos.utfpr.edu.br}}

\begin{document}
\maketitle

\begin{abstract}
We study driver behavior classification using the UAH-DriveSet with smartphone IMU/GPS.
We compare Random Forest, Logistic Regression and Linear SVM under leave-one-driver-out (LODO).
With 10\,s windows and 50\% overlap, Random Forest achieves macro-F1 of 0.6499.
To increase recall for drowsy events on motorways, we calibrate a per-route threshold ($\tau=0.30$),
raising recall from 0.4626 to 0.6395 with negligible global macro-F1 change (0.644$\to$0.643).
\end{abstract}

\section{Introdução}
Motivação (segurança viária), objetivo, contribuição (LODO por motorista, calibração por rota).

\section{Base de dados}
UAH-DriveSet; janela 10\,s (50\% overlap). Contagens: aggressive 1593, drowsy 1998, normal 2637.
Rotas: \emph{motorway}, \emph{secondary}. Como o CSV de sessões foi montado.

\section{Pré-processamento}
Imputação, padronização; PCA p/ lineares. Features: jerk, p95, lf\_ratio. Compat.: \texttt{np.trapezoid}.

\section{Modelos e protocolo}
LODO por motorista. RF (class\_weight=balanced), LogReg+PCA, LinSVC+PCA. Métricas: accuracy, macro-F1, PR-AUC.

\section{Resultados}
\begin{table}[h]\centering
\begin{tabular}{lcc}\toprule
Modelo & Accuracy & Macro-F1\\\midrule
Random Forest & \textbf{0.6601} & \textbf{0.6499}\\
Logistic Regression & 0.5658 & 0.5636\\
Linear SVM & 0.5639 & 0.5572\\\bottomrule
\end{tabular}
\caption{LODO (10\,s).}\label{tab:models}\end{table}

\begin{figure}[h]\centering
\includegraphics[width=.32\linewidth]{figs/prcurve_rf_driver_aggressive.png}
\includegraphics[width=.32\linewidth]{figs/prcurve_rf_driver_drowsy.png}
\includegraphics[width=.32\linewidth]{figs/prcurve_rf_driver_normal.png}
\caption{PR-curves por classe (AP: Agg 0.7621; Drowsy 0.6704; Normal 0.6627).}\label{fig:pr}
\end{figure}

\begin{figure}[h]\centering
\includegraphics[width=.55\linewidth]{figs/fig_rf_driver_f1_5s_10s.png}
\caption{Macro-F1: 10\,s $>$ 5\,s (RF).}\label{fig:win}
\end{figure}

Calibração em \emph{motorway} ($\tau{=}0.30$):
prec(drowsy)=0.5713, rec(drowsy)=0.6395, macro-F1 global 0.6433 (\emph{vs} 0.6444).

\section{Discussão e conclusão}
10\,s ajuda; RF $>$ lineares; calibração por rota aumenta recall de drowsy com custo pequeno em precisão.
Trabalho futuro: calibração por motorista, GBMs, calibration/interpretabilidade.

\bibliographystyle{sbc}\bibliography{refs}
\end{document}
